% Options for packages loaded elsewhere
\PassOptionsToPackage{unicode}{hyperref}
\PassOptionsToPackage{hyphens}{url}
%
\documentclass[
]{article}
\usepackage{lmodern}
\usepackage{amssymb,amsmath}
\usepackage{ifxetex,ifluatex}
\ifnum 0\ifxetex 1\fi\ifluatex 1\fi=0 % if pdftex
  \usepackage[T1]{fontenc}
  \usepackage[utf8]{inputenc}
  \usepackage{textcomp} % provide euro and other symbols
\else % if luatex or xetex
  \usepackage{unicode-math}
  \defaultfontfeatures{Scale=MatchLowercase}
  \defaultfontfeatures[\rmfamily]{Ligatures=TeX,Scale=1}
\fi
% Use upquote if available, for straight quotes in verbatim environments
\IfFileExists{upquote.sty}{\usepackage{upquote}}{}
\IfFileExists{microtype.sty}{% use microtype if available
  \usepackage[]{microtype}
  \UseMicrotypeSet[protrusion]{basicmath} % disable protrusion for tt fonts
}{}
\makeatletter
\@ifundefined{KOMAClassName}{% if non-KOMA class
  \IfFileExists{parskip.sty}{%
    \usepackage{parskip}
  }{% else
    \setlength{\parindent}{0pt}
    \setlength{\parskip}{6pt plus 2pt minus 1pt}}
}{% if KOMA class
  \KOMAoptions{parskip=half}}
\makeatother
\usepackage{xcolor}
\IfFileExists{xurl.sty}{\usepackage{xurl}}{} % add URL line breaks if available
\IfFileExists{bookmark.sty}{\usepackage{bookmark}}{\usepackage{hyperref}}
\hypersetup{
  pdftitle={King County House Price Sales Final Project},
  hidelinks,
  pdfcreator={LaTeX via pandoc}}
\urlstyle{same} % disable monospaced font for URLs
\usepackage[margin=1in]{geometry}
\usepackage{color}
\usepackage{fancyvrb}
\newcommand{\VerbBar}{|}
\newcommand{\VERB}{\Verb[commandchars=\\\{\}]}
\DefineVerbatimEnvironment{Highlighting}{Verbatim}{commandchars=\\\{\}}
% Add ',fontsize=\small' for more characters per line
\usepackage{framed}
\definecolor{shadecolor}{RGB}{248,248,248}
\newenvironment{Shaded}{\begin{snugshade}}{\end{snugshade}}
\newcommand{\AlertTok}[1]{\textcolor[rgb]{0.94,0.16,0.16}{#1}}
\newcommand{\AnnotationTok}[1]{\textcolor[rgb]{0.56,0.35,0.01}{\textbf{\textit{#1}}}}
\newcommand{\AttributeTok}[1]{\textcolor[rgb]{0.77,0.63,0.00}{#1}}
\newcommand{\BaseNTok}[1]{\textcolor[rgb]{0.00,0.00,0.81}{#1}}
\newcommand{\BuiltInTok}[1]{#1}
\newcommand{\CharTok}[1]{\textcolor[rgb]{0.31,0.60,0.02}{#1}}
\newcommand{\CommentTok}[1]{\textcolor[rgb]{0.56,0.35,0.01}{\textit{#1}}}
\newcommand{\CommentVarTok}[1]{\textcolor[rgb]{0.56,0.35,0.01}{\textbf{\textit{#1}}}}
\newcommand{\ConstantTok}[1]{\textcolor[rgb]{0.00,0.00,0.00}{#1}}
\newcommand{\ControlFlowTok}[1]{\textcolor[rgb]{0.13,0.29,0.53}{\textbf{#1}}}
\newcommand{\DataTypeTok}[1]{\textcolor[rgb]{0.13,0.29,0.53}{#1}}
\newcommand{\DecValTok}[1]{\textcolor[rgb]{0.00,0.00,0.81}{#1}}
\newcommand{\DocumentationTok}[1]{\textcolor[rgb]{0.56,0.35,0.01}{\textbf{\textit{#1}}}}
\newcommand{\ErrorTok}[1]{\textcolor[rgb]{0.64,0.00,0.00}{\textbf{#1}}}
\newcommand{\ExtensionTok}[1]{#1}
\newcommand{\FloatTok}[1]{\textcolor[rgb]{0.00,0.00,0.81}{#1}}
\newcommand{\FunctionTok}[1]{\textcolor[rgb]{0.00,0.00,0.00}{#1}}
\newcommand{\ImportTok}[1]{#1}
\newcommand{\InformationTok}[1]{\textcolor[rgb]{0.56,0.35,0.01}{\textbf{\textit{#1}}}}
\newcommand{\KeywordTok}[1]{\textcolor[rgb]{0.13,0.29,0.53}{\textbf{#1}}}
\newcommand{\NormalTok}[1]{#1}
\newcommand{\OperatorTok}[1]{\textcolor[rgb]{0.81,0.36,0.00}{\textbf{#1}}}
\newcommand{\OtherTok}[1]{\textcolor[rgb]{0.56,0.35,0.01}{#1}}
\newcommand{\PreprocessorTok}[1]{\textcolor[rgb]{0.56,0.35,0.01}{\textit{#1}}}
\newcommand{\RegionMarkerTok}[1]{#1}
\newcommand{\SpecialCharTok}[1]{\textcolor[rgb]{0.00,0.00,0.00}{#1}}
\newcommand{\SpecialStringTok}[1]{\textcolor[rgb]{0.31,0.60,0.02}{#1}}
\newcommand{\StringTok}[1]{\textcolor[rgb]{0.31,0.60,0.02}{#1}}
\newcommand{\VariableTok}[1]{\textcolor[rgb]{0.00,0.00,0.00}{#1}}
\newcommand{\VerbatimStringTok}[1]{\textcolor[rgb]{0.31,0.60,0.02}{#1}}
\newcommand{\WarningTok}[1]{\textcolor[rgb]{0.56,0.35,0.01}{\textbf{\textit{#1}}}}
\usepackage{graphicx,grffile}
\makeatletter
\def\maxwidth{\ifdim\Gin@nat@width>\linewidth\linewidth\else\Gin@nat@width\fi}
\def\maxheight{\ifdim\Gin@nat@height>\textheight\textheight\else\Gin@nat@height\fi}
\makeatother
% Scale images if necessary, so that they will not overflow the page
% margins by default, and it is still possible to overwrite the defaults
% using explicit options in \includegraphics[width, height, ...]{}
\setkeys{Gin}{width=\maxwidth,height=\maxheight,keepaspectratio}
% Set default figure placement to htbp
\makeatletter
\def\fps@figure{htbp}
\makeatother
\setlength{\emergencystretch}{3em} % prevent overfull lines
\providecommand{\tightlist}{%
  \setlength{\itemsep}{0pt}\setlength{\parskip}{0pt}}
\setcounter{secnumdepth}{-\maxdimen} % remove section numbering

\title{King County House Price Sales Final Project}
\author{}
\date{\vspace{-2.5em}}

\begin{document}
\maketitle

~

\hypertarget{by-lahari-kuchibhotla-stuti-sanghavi-tanmayi-varansi-and-sanjana-ramakandth}{%
\subsubsection{By Lahari Kuchibhotla, Stuti Sanghavi, Tanmayi Varansi
and Sanjana
Ramakandth}\label{by-lahari-kuchibhotla-stuti-sanghavi-tanmayi-varansi-and-sanjana-ramakandth}}

~

\hypertarget{abstract}{%
\subsection{Abstract}\label{abstract}}

This report examines the causal effect house size along with a variety
of amenities has on house price evidenced by data collected for house
sales in King County, USA between 2014 and 2015. After utilizing
multivariable regression for the analysis, the conclusion drawn was that
as house size increases and amenities such as having a waterfront, a
nicer view, renovated home or better construction all play a role in
increasing the price of a house.

~

\hypertarget{introduction}{%
\subsection{Introduction}\label{introduction}}

The Housing Sales dataset from Kaggle is about homes sold in King
County, Seattle. The observations were made between May 2014 and May
2015. The data contains around 21K rows and 21 variables about various
features of the home as well as the price the home was sold at and the
location.

~

\hypertarget{data-cleaning-and-manipulation}{%
\paragraph{Data Cleaning and
Manipulation}\label{data-cleaning-and-manipulation}}

\begin{Shaded}
\begin{Highlighting}[]
\CommentTok{#Reading the csv file}
\NormalTok{data <-}\StringTok{ }\KeywordTok{read.csv}\NormalTok{(}\StringTok{"kc_house_data.csv"}\NormalTok{)}
\KeywordTok{head}\NormalTok{(data)}
\end{Highlighting}
\end{Shaded}

\begin{verbatim}
##           id            date   price bedrooms bathrooms sqft_living sqft_lot
## 1 7129300520 20141013T000000  221900        3      1.00        1180     5650
## 2 6414100192 20141209T000000  538000        3      2.25        2570     7242
## 3 5631500400 20150225T000000  180000        2      1.00         770    10000
## 4 2487200875 20141209T000000  604000        4      3.00        1960     5000
## 5 1954400510 20150218T000000  510000        3      2.00        1680     8080
## 6 7237550310 20140512T000000 1225000        4      4.50        5420   101930
##   floors waterfront view condition grade sqft_above sqft_basement yr_built
## 1      1          0    0         3     7       1180             0     1955
## 2      2          0    0         3     7       2170           400     1951
## 3      1          0    0         3     6        770             0     1933
## 4      1          0    0         5     7       1050           910     1965
## 5      1          0    0         3     8       1680             0     1987
## 6      1          0    0         3    11       3890          1530     2001
##   yr_renovated zipcode     lat     long sqft_living15 sqft_lot15
## 1            0   98178 47.5112 -122.257          1340       5650
## 2         1991   98125 47.7210 -122.319          1690       7639
## 3            0   98028 47.7379 -122.233          2720       8062
## 4            0   98136 47.5208 -122.393          1360       5000
## 5            0   98074 47.6168 -122.045          1800       7503
## 6            0   98053 47.6561 -122.005          4760     101930
\end{verbatim}

~

\hypertarget{basic-descriptive-statistics}{%
\paragraph{Basic Descriptive
Statistics}\label{basic-descriptive-statistics}}

\begin{Shaded}
\begin{Highlighting}[]
\CommentTok{#Getting descriptive statistics for data}
\NormalTok{res <-}\StringTok{ }\KeywordTok{stat.desc}\NormalTok{(data)}
\KeywordTok{print}\NormalTok{(res)}
\end{Highlighting}
\end{Shaded}

\begin{verbatim}
##                        id date        price     bedrooms    bathrooms
## nbr.val      2.161300e+04   NA 2.161300e+04 2.161300e+04 2.161300e+04
## nbr.null     0.000000e+00   NA 0.000000e+00 1.300000e+01 1.000000e+01
## nbr.na       0.000000e+00   NA 0.000000e+00 0.000000e+00 0.000000e+00
## min          1.000102e+06   NA 7.500000e+04 0.000000e+00 0.000000e+00
## max          9.900000e+09   NA 7.700000e+06 3.300000e+01 8.000000e+00
## range        9.899000e+09   NA 7.625000e+06 3.300000e+01 8.000000e+00
## sum          9.899406e+13   NA 1.167293e+10 7.285400e+04 4.570625e+04
## median       3.904930e+09   NA 4.500000e+05 3.000000e+00 2.250000e+00
## mean         4.580302e+09   NA 5.400881e+05 3.370842e+00 2.114757e+00
## SE.mean      1.956666e+07   NA 2.497233e+03 6.326366e-03 5.238720e-03
## CI.mean.0.95 3.835210e+07   NA 4.894760e+03 1.240014e-02 1.026828e-02
## var          8.274629e+18   NA 1.347824e+11 8.650150e-01 5.931513e-01
## std.dev      2.876566e+09   NA 3.671272e+05 9.300618e-01 7.701632e-01
## coef.var     6.280297e-01   NA 6.797542e-01 2.759138e-01 3.641851e-01
##               sqft_living     sqft_lot       floors   waterfront         view
## nbr.val      2.161300e+04 2.161300e+04 2.161300e+04 2.161300e+04 2.161300e+04
## nbr.null     0.000000e+00 0.000000e+00 0.000000e+00 2.145000e+04 1.948900e+04
## nbr.na       0.000000e+00 0.000000e+00 0.000000e+00 0.000000e+00 0.000000e+00
## min          2.900000e+02 5.200000e+02 1.000000e+00 0.000000e+00 0.000000e+00
## max          1.354000e+04 1.651359e+06 3.500000e+00 1.000000e+00 4.000000e+00
## range        1.325000e+04 1.650839e+06 2.500000e+00 1.000000e+00 4.000000e+00
## sum          4.495287e+07 3.265069e+08 3.229650e+04 1.630000e+02 5.064000e+03
## median       1.910000e+03 7.618000e+03 1.500000e+00 0.000000e+00 0.000000e+00
## mean         2.079900e+03 1.510697e+04 1.494309e+00 7.541757e-03 2.343034e-01
## SE.mean      6.247319e+00 2.817461e+02 3.673054e-03 5.884979e-04 5.212562e-03
## CI.mean.0.95 1.224521e+01 5.522432e+02 7.199457e-03 1.153499e-03 1.021701e-02
## var          8.435337e+05 1.715659e+09 2.915880e-01 7.485226e-03 5.872426e-01
## std.dev      9.184409e+02 4.142051e+04 5.399889e-01 8.651720e-02 7.663176e-01
## coef.var     4.415794e-01 2.741815e+00 3.613636e-01 1.147176e+01 3.270620e+00
##                 condition        grade   sqft_above sqft_basement     yr_built
## nbr.val      2.161300e+04 2.161300e+04 2.161300e+04  2.161300e+04 2.161300e+04
## nbr.null     0.000000e+00 0.000000e+00 0.000000e+00  1.312600e+04 0.000000e+00
## nbr.na       0.000000e+00 0.000000e+00 0.000000e+00  0.000000e+00 0.000000e+00
## min          1.000000e+00 1.000000e+00 2.900000e+02  0.000000e+00 1.900000e+03
## max          5.000000e+00 1.300000e+01 9.410000e+03  4.820000e+03 2.015000e+03
## range        4.000000e+00 1.200000e+01 9.120000e+03  4.820000e+03 1.150000e+02
## sum          7.368800e+04 1.654880e+05 3.865249e+07  6.300385e+06 4.259933e+07
## median       3.000000e+00 7.000000e+00 1.560000e+03  0.000000e+00 1.975000e+03
## mean         3.409430e+00 7.656873e+00 1.788391e+03  2.915090e+02 1.971005e+03
## SE.mean      4.426414e-03 7.995578e-03 5.632751e+00  3.010436e+00 1.998006e-01
## CI.mean.0.95 8.676097e-03 1.567192e-02 1.104061e+01  5.900677e+00 3.916240e-01
## var          4.234665e-01 1.381703e+00 6.857347e+05  1.958727e+05 8.627973e+02
## std.dev      6.507430e-01 1.175459e+00 8.280910e+02  4.425750e+02 2.937341e+01
## coef.var     1.908657e-01 1.535168e-01 4.630370e-01  1.518221e+00 1.490276e-02
##              yr_renovated      zipcode          lat          long sqft_living15
## nbr.val      2.161300e+04 2.161300e+04 2.161300e+04  2.161300e+04  2.161300e+04
## nbr.null     2.069900e+04 0.000000e+00 0.000000e+00  0.000000e+00  0.000000e+00
## nbr.na       0.000000e+00 0.000000e+00 0.000000e+00  0.000000e+00  0.000000e+00
## min          0.000000e+00 9.800100e+04 4.715590e+01 -1.225190e+02  3.990000e+02
## max          2.015000e+03 9.819900e+04 4.777760e+01 -1.213150e+02  6.210000e+03
## range        2.015000e+03 1.980000e+02 6.217000e-01  1.204000e+00  5.811000e+03
## sum          1.824186e+06 2.119759e+09 1.027915e+06 -2.641409e+06  4.293536e+07
## median       0.000000e+00 9.806500e+04 4.757180e+01 -1.222300e+02  1.840000e+03
## mean         8.440226e+01 9.807794e+04 4.756005e+01 -1.222139e+02  1.986552e+03
## SE.mean      2.732259e+00 3.639461e-01 9.425230e-04  9.579273e-04  4.662094e+00
## CI.mean.0.95 5.355429e+00 7.133612e-01 1.847415e-03  1.877608e-03  9.138049e+00
## var          1.613462e+05 2.862788e+03 1.919990e-02  1.983262e-02  4.697612e+05
## std.dev      4.016792e+02 5.350503e+01 1.385637e-01  1.408283e-01  6.853913e+02
## coef.var     4.759105e+00 5.455358e-04 2.913447e-03 -1.152310e-03  3.450155e-01
##                sqft_lot15
## nbr.val      2.161300e+04
## nbr.null     0.000000e+00
## nbr.na       0.000000e+00
## min          6.510000e+02
## max          8.712000e+05
## range        8.705490e+05
## sum          2.759646e+08
## median       7.620000e+03
## mean         1.276846e+04
## SE.mean      1.857255e+02
## CI.mean.0.95 3.640357e+02
## var          7.455182e+08
## std.dev      2.730418e+04
## coef.var     2.138409e+00
\end{verbatim}

The above table shows the descriptive statistics for the entire dataset.
Majority of the outliers were linked to having a nice view or a
waterfront so they were not removed from the model. The one outlier that
was removed was a house that had 33 bathrooms and only 1.75 bathrooms.
This does not make sense as the house was only 1600 sqft.

~

\hypertarget{data-cleaning}{%
\paragraph{Data Cleaning}\label{data-cleaning}}

Before running our models we cleaned the dataset.

\begin{itemize}
\item
  As price was on a much larger scale than the other variables, we
  scaled it down by 1000 so that when the variables were plotted
  together it would be easier to analyze.
\item
  For easier interpretation of results, we added another column called
  \textbf{age\_of\_house} where we subtracted the year build from the
  current year.
\item
  For easier interpretation of results, we converted the yr\_renovated
  column into a binary column and named it \textbf{renovated\_factor}.
\item
  As discussed above, removed the outlier that had 33 bathrooms and only
  1.75 bathrooms.
\end{itemize}

\begin{Shaded}
\begin{Highlighting}[]
\CommentTok{# Scaling down the price variable for better readability}
\NormalTok{data}\OperatorTok{$}\NormalTok{price =}\StringTok{ }\NormalTok{data}\OperatorTok{$}\NormalTok{price}\OperatorTok{/}\DecValTok{1000}

\CommentTok{# Adding a new column "age_of_house"}
\NormalTok{data}\OperatorTok{$}\NormalTok{age_of_house <-}\StringTok{ }\KeywordTok{as.integer}\NormalTok{(}\KeywordTok{format}\NormalTok{(}\KeywordTok{Sys.Date}\NormalTok{(), }\StringTok{"%Y"}\NormalTok{)) }\OperatorTok{-}\StringTok{ }\NormalTok{data}\OperatorTok{$}\NormalTok{yr_built}
\NormalTok{data <-}\StringTok{ }\NormalTok{data[,}\OperatorTok{-}\KeywordTok{c}\NormalTok{(}\DecValTok{1}\NormalTok{,}\DecValTok{2}\NormalTok{,}\DecValTok{15}\NormalTok{)]}

\CommentTok{#Creating a binary column for yr_rennovated where:}
\CommentTok{# 1 = House is renovated }
\CommentTok{# 0 = House is not renovated}
\NormalTok{data}\OperatorTok{$}\NormalTok{renovated_factor=}\KeywordTok{ifelse}\NormalTok{(data}\OperatorTok{$}\NormalTok{yr_renovated }\OperatorTok{!=}\StringTok{ }\DecValTok{0}\NormalTok{, }\DecValTok{1}\NormalTok{, }\DecValTok{0}\NormalTok{)}

\CommentTok{#Removing the outlier}
\NormalTok{data <-}\StringTok{ }\KeywordTok{subset}\NormalTok{(data, bedrooms }\OperatorTok{<}\DecValTok{30}\NormalTok{)}

\KeywordTok{head}\NormalTok{(data)}
\end{Highlighting}
\end{Shaded}

\begin{verbatim}
##    price bedrooms bathrooms sqft_living sqft_lot floors waterfront view
## 1  221.9        3      1.00        1180     5650      1          0    0
## 2  538.0        3      2.25        2570     7242      2          0    0
## 3  180.0        2      1.00         770    10000      1          0    0
## 4  604.0        4      3.00        1960     5000      1          0    0
## 5  510.0        3      2.00        1680     8080      1          0    0
## 6 1225.0        4      4.50        5420   101930      1          0    0
##   condition grade sqft_above sqft_basement yr_renovated zipcode     lat
## 1         3     7       1180             0            0   98178 47.5112
## 2         3     7       2170           400         1991   98125 47.7210
## 3         3     6        770             0            0   98028 47.7379
## 4         5     7       1050           910            0   98136 47.5208
## 5         3     8       1680             0            0   98074 47.6168
## 6         3    11       3890          1530            0   98053 47.6561
##       long sqft_living15 sqft_lot15 age_of_house renovated_factor
## 1 -122.257          1340       5650           65                0
## 2 -122.319          1690       7639           69                1
## 3 -122.233          2720       8062           87                0
## 4 -122.393          1360       5000           55                0
## 5 -122.045          1800       7503           33                0
## 6 -122.005          4760     101930           19                0
\end{verbatim}

~

\hypertarget{variables-used-in-our-analysis}{%
\paragraph{Variables used in our
analysis:}\label{variables-used-in-our-analysis}}

\begin{enumerate}
\def\labelenumi{\arabic{enumi}.}
\item
  price - Price of each home sold (outcome variable)
\item
  bedrooms - Number of bedrooms
\item
  bathrooms - Number of bathrooms, where .5 accounts for a room with a
  toilet but no shower
\item
  sqft\_living - Square footage of interior living space of the house
\item
  sqft\_lot - Square footage of the land space
\item
  floors - Number of floors
\item
  waterfront - A dummy variable for whether the apartment was
  overlooking the waterfront or not
\item
  view - An index from 0 to 4 of how good the view of the property was
\item
  condition - An index from 1 to 5 on the condition of the apartment,
\item
  grade - An index from 1 to 13, where 1-6 falls short of building
  construction and design, 7 has an average level of construction and
  design, and 11-13 have a high quality level of construction and
  design.
\item
  sqft\_above - The square footage of the interior housing space that is
  above ground level
\item
  sqft\_basement - The square footage of the interior housing space that
  is below ground level
\item
  age\_of\_house - Tells us how old the house is
\item
  yr\_renovated - The year of the house's last renovation
\item
  zipcode - What zipcode area the house is in
\end{enumerate}

~

\hypertarget{exploring-the-data}{%
\subsection{Exploring the data}\label{exploring-the-data}}

~

\hypertarget{basic-distribution-plot}{%
\paragraph{Basic Distribution plot}\label{basic-distribution-plot}}

\begin{Shaded}
\begin{Highlighting}[]
\CommentTok{#Distribution of the outcome variable (price)}

\CommentTok{# We can see from the graph that the data is skewed.}
\KeywordTok{hist}\NormalTok{(data}\OperatorTok{$}\NormalTok{price,}\DataTypeTok{border=}\StringTok{"red"}\NormalTok{, }\DataTypeTok{xlab=}\StringTok{"price"}\NormalTok{, }\DataTypeTok{main =} \StringTok{"Distribution of price (Y)"}\NormalTok{)}
\end{Highlighting}
\end{Shaded}

\includegraphics{Final_Presentation_Report_Group_3_files/figure-latex/unnamed-chunk-5-1.pdf}

~

\hypertarget{correlations-between-the-variables}{%
\paragraph{Correlations between the
variables}\label{correlations-between-the-variables}}

Positive correlations are displayed in blue and negative correlations in
red color. Color intensity and the size of the circle are proportional
to the correlation coefficients. In the right side of the correlogram,
the legend color shows the correlation coefficients and the
corresponding colors.

\begin{Shaded}
\begin{Highlighting}[]
\CommentTok{# Correlation plot}
\NormalTok{data[, }\DecValTok{3}\OperatorTok{:}\DecValTok{20}\NormalTok{] }\OperatorTok\StringTok{ }
\StringTok{  }\NormalTok{dplyr}\OperatorTok{::}\KeywordTok{select}\NormalTok{(}\OperatorTok{-}\NormalTok{zipcode) }\OperatorTok\StringTok{ }\KeywordTok{cor}\NormalTok{() }\OperatorTok\StringTok{ }
\StringTok{  }\NormalTok{corrplot}\OperatorTok{::}\KeywordTok{corrplot}\NormalTok{(}\DataTypeTok{type =} \StringTok{"lower"}\NormalTok{, }\DataTypeTok{order =} \StringTok{"hclust"}\NormalTok{, }\DataTypeTok{tl.col =} \StringTok{"grey30"}\NormalTok{, }\DataTypeTok{tl.cex =} \FloatTok{0.9}\NormalTok{)}
\end{Highlighting}
\end{Shaded}

\includegraphics{Final_Presentation_Report_Group_3_files/figure-latex/unnamed-chunk-6-1.pdf}

~

\hypertarget{building-our-base-specification-model}{%
\subsection{Building our base specification
model}\label{building-our-base-specification-model}}

~

\hypertarget{regressing-house-price-on-sqft_living}{%
\paragraph{Regressing house price on
sqft\_living}\label{regressing-house-price-on-sqft_living}}

\begin{Shaded}
\begin{Highlighting}[]
\CommentTok{#Base specification model between our dependent variable and variable of interest.}
\NormalTok{sqft_living <-}\StringTok{ }\KeywordTok{lm}\NormalTok{(price }\OperatorTok{~}\StringTok{ }\NormalTok{sqft_living, }\DataTypeTok{data =}\NormalTok{ data)}

\KeywordTok{summary}\NormalTok{(sqft_living)}
\end{Highlighting}
\end{Shaded}

\begin{verbatim}
## 
## Call:
## lm(formula = price ~ sqft_living, data = data)
## 
## Residuals:
##     Min      1Q  Median      3Q     Max 
## -1476.1  -147.5   -24.1   106.2  4362.0 
## 
## Coefficients:
##               Estimate Std. Error t value Pr(>|t|)    
## (Intercept) -43.603353   4.402789  -9.904   <2e-16 ***
## sqft_living   0.280629   0.001936 144.922   <2e-16 ***
## ---
## Signif. codes:  0 '***' 0.001 '**' 0.01 '*' 0.05 '.' 0.1 ' ' 1
## 
## Residual standard error: 261.5 on 21610 degrees of freedom
## Multiple R-squared:  0.4929, Adjusted R-squared:  0.4928 
## F-statistic: 2.1e+04 on 1 and 21610 DF,  p-value: < 2.2e-16
\end{verbatim}

From the above results, we can see that sqft\_living is positively
correlated with the the price of the house. i.e.~If the sqft\_living
increases by 1 foot, the price increases by \$280.

~

\hypertarget{plotting-price-vs-sqft_living}{%
\paragraph{Plotting price vs
sqft\_living}\label{plotting-price-vs-sqft_living}}

\begin{Shaded}
\begin{Highlighting}[]
\CommentTok{# Plot 1: plotting price vs sqft_living}
\KeywordTok{ggplot}\NormalTok{(data, }\KeywordTok{aes}\NormalTok{(}\DataTypeTok{x=}\NormalTok{sqft_living, }\DataTypeTok{y=}\NormalTok{price)) }\OperatorTok{+}\StringTok{ }\KeywordTok{geom_point}\NormalTok{(}\DataTypeTok{col=}\StringTok{"blue"}\NormalTok{) }\OperatorTok{+}\StringTok{ }
\KeywordTok{labs}\NormalTok{(}\DataTypeTok{title =} \StringTok{"Price vs Sqft_living"}\NormalTok{, }\DataTypeTok{x =} \StringTok{"sqft_living"}\NormalTok{, }\DataTypeTok{y =} \StringTok{"Price"}\NormalTok{) }\OperatorTok{+}
\KeywordTok{stat_smooth}\NormalTok{(}\DataTypeTok{method =} \StringTok{"lm"}\NormalTok{, }\DataTypeTok{col =} \StringTok{"red"}\NormalTok{, }\DataTypeTok{se=}\OtherTok{FALSE}\NormalTok{) }\OperatorTok{+}\StringTok{ }\KeywordTok{theme}\NormalTok{(}\DataTypeTok{plot.title =} \KeywordTok{element_text}\NormalTok{(}\DataTypeTok{hjust =} \FloatTok{0.5}\NormalTok{))}
\end{Highlighting}
\end{Shaded}

\includegraphics{Final_Presentation_Report_Group_3_files/figure-latex/unnamed-chunk-8-1.pdf}

~

\hypertarget{plotting-the-sqft_living-and-residuals}{%
\paragraph{Plotting the sqft\_living and
residuals}\label{plotting-the-sqft_living-and-residuals}}

Here, we notice that the variance of the residuals increases with
sqft\_living

\begin{Shaded}
\begin{Highlighting}[]
\CommentTok{#Plot 2: plotting the residuals}
\NormalTok{df_resid =}\StringTok{ }\NormalTok{data}

\NormalTok{df_resid}\OperatorTok{$}\NormalTok{resid<-}\KeywordTok{resid}\NormalTok{(sqft_living)}

\KeywordTok{ggplot}\NormalTok{(df_resid, }\KeywordTok{aes}\NormalTok{(}\DataTypeTok{x=}\NormalTok{sqft_living, }\DataTypeTok{y=}\NormalTok{resid)) }\OperatorTok{+}\StringTok{ }\KeywordTok{geom_point}\NormalTok{(}\DataTypeTok{col=}\StringTok{"blue"}\NormalTok{) }\OperatorTok{+}\StringTok{ }
\KeywordTok{labs}\NormalTok{(}\DataTypeTok{title =} \StringTok{"sqft_living and Residuals"}\NormalTok{, }\DataTypeTok{x =} \StringTok{"sqft_living"}\NormalTok{, }\DataTypeTok{y =} \StringTok{"Residuals"}\NormalTok{)  }\OperatorTok{+}\StringTok{ }\KeywordTok{theme}\NormalTok{(}\DataTypeTok{plot.title =} \KeywordTok{element_text}\NormalTok{(}\DataTypeTok{hjust =} \FloatTok{0.5}\NormalTok{))}
\end{Highlighting}
\end{Shaded}

\includegraphics{Final_Presentation_Report_Group_3_files/figure-latex/unnamed-chunk-9-1.pdf}

From the Plot 1 and Plot 2, we see that the errors are heteroskadastic.
Therefore we need to correct for standard errors.

To do that, we use CSE function to calculate heteroskedastic-robust
standard errors.

\begin{Shaded}
\begin{Highlighting}[]
\CommentTok{# CSE function is used to correct the standard errors}
\NormalTok{cse=}\ControlFlowTok{function}\NormalTok{(reg) \{}

\NormalTok{        rob=}\KeywordTok{sqrt}\NormalTok{(}\KeywordTok{diag}\NormalTok{(}\KeywordTok{vcovHC}\NormalTok{(reg, }\DataTypeTok{type=}\StringTok{"HC2"}\NormalTok{)))}

        \KeywordTok{return}\NormalTok{(rob)}
\NormalTok{\}}
\end{Highlighting}
\end{Shaded}

~

\hypertarget{looking-at-potential-control-variables}{%
\subsection{Looking at Potential Control
Variables}\label{looking-at-potential-control-variables}}

After researching the relevant literature , we notice that our dependent
variable (house price) is broadly dependent on different feature
categories as follows:

\begin{enumerate}
\def\labelenumi{\arabic{enumi}.}
\item
  Size of the house : We know that size of the house plays an important
  factor in determining whether or not the house prices increase and we
  expect that as the size of the house increases, the house price should
  increase. The size of the house is captured by various variables in
  the dataset such as bedrooms, bathrooms, sqft\_living, floors,
  sqft\_lot, sqft\_above, sqft\_basement.
\item
  Location: The Location of the house definitely plays an important role
  in determining the house price. We expect that houses having better
  location and safer neighbourhoods have higher prices as compared to
  unsafe locations and similarly, houses near water bodies are more
  expensive as compared to others. The variables in the dataset
  indicative of the location of the house include, waterfront, zipcode,
  lat and long. This is clearly indicated by the graph below where we
  see that the prices are higher for properties which are located near a
  waterfront / properties that may be in a safe neighbourhood as
  compared to others.
\end{enumerate}

\begin{Shaded}
\begin{Highlighting}[]
\CommentTok{#Estimating home density and price}
\NormalTok{map <-}\StringTok{ }\KeywordTok{qmplot}\NormalTok{(long, lat, }\DataTypeTok{data =}\NormalTok{ data, }\DataTypeTok{geom =} \StringTok{"blank"}\NormalTok{,}
       \DataTypeTok{maptype =} \StringTok{"toner-lite"}\NormalTok{) }\OperatorTok{+}
\StringTok{  }\KeywordTok{stat_density_2d}\NormalTok{(}\KeywordTok{aes}\NormalTok{(}\DataTypeTok{fill =}\NormalTok{ price), }
                  \DataTypeTok{geom =} \StringTok{"polygon"}\NormalTok{, }\DataTypeTok{alpha =} \FloatTok{.5}\NormalTok{, }\DataTypeTok{color =} \OtherTok{NA}\NormalTok{) }\OperatorTok{+}
\StringTok{  }\KeywordTok{scale_fill_gradient2}\NormalTok{(}\DataTypeTok{low =} \StringTok{"steelblue"}\NormalTok{, }\DataTypeTok{mid =} \StringTok{"seagreen4"}\NormalTok{, }
                        \DataTypeTok{high =} \StringTok{"indianred"}\NormalTok{, }\DataTypeTok{midpoint =} \DecValTok{7}\NormalTok{) }\OperatorTok{+}
\StringTok{  }\KeywordTok{labs}\NormalTok{(}\DataTypeTok{title =} \StringTok{"Estimated home density"}\NormalTok{)}
\NormalTok{map}
\end{Highlighting}
\end{Shaded}

\includegraphics{Final_Presentation_Report_Group_3_files/figure-latex/unnamed-chunk-11-1.pdf}

\begin{enumerate}
\def\labelenumi{\arabic{enumi}.}
\setcounter{enumi}{2}
\item
  View : If a particular house has a view or not. We expect that houses
  with better views have higher prices. A house having a city view,
  ocean view, mountain view etc have higher prices as compared to houses
  which are more inland. The variable capturing that in the dataset is
  view.
\item
  Conditions: If the house is in a better condition, it would require
  lesser maintainence work and the house prices would generally be
  higher for prices with better condition. If the house has better
  construction, meaning it has better plumbing, roofing or newer
  appliances the house would be more expensive. Similarily if the house
  had a lower Grade the house would be less expensive.The variables
  which talk about the conditions of the home in our dataset are,
  condition and grade
\item
  Age of the house: We would expect that if the age of the house
  increases, i.e.~if its an old house, the price of the house should
  decrease. Similarly, if it's a new construction / newly renovated
  home, the house price should be higher. The variables indicative of
  the age of the house in our dataset are, age\_of\_house and
  yr\_renovated variable.
\end{enumerate}

Other factors such as economic growth, supply and demand and interest
rates all play a role in house prices. However our dataset does not
include these factors so we will use the above mentioned variables to
see which of these variables help us explain the effect of sqft\_living
on house price.

~

\hypertarget{research-questionhypothesis}{%
\subsection{Research
Question/Hypothesis:}\label{research-questionhypothesis}}

\begin{itemize}
\item
  \textbf{Research Question:} What is the causal effect of sqft\_living
  on house price?
\item
  \textbf{Null Hypothesis:} There is no difference in house price when
  sqft\_living increases and amenities such as having view, waterfront,
  renovation and grade are added.
\item
  \textbf{Alternative Hypothesis:} Sqft\_living, having a waterfront,
  view, Renovation and/or grade impact the price of the house.
\item
  \textbf{Variable of Interest:} Sqft\_living
\item
  \textbf{Dependent Variable:} House Price
\item
  \textbf{Control Variables:} Waterfront, View, Renovation and Grade
\end{itemize}

~

\hypertarget{nested-method-analysis}{%
\subsection{Nested Method Analysis}\label{nested-method-analysis}}

~

For the variables that fall under each of the categories mentioned
above, we used a nested approach to determine whether they should be
included in the final model or not. In the nested approach we run two
variable regression models, wherein we see how adding one variable at a
time to the base model affects the newly created model and how that
compares to the base specification. For all the variables falling under
one category, we plan to shortlist and use only one of the variables as
control variables. The variable selected from each category would be the
one that has the most effect or impact on house price when added in
conjunction with the variable of interest. We plan to use only one
variable from each category and not more than that because we might face
an issue of multicollinearity. However we also plan to select atleast
one variable from each category of size, location, view, condition and
age to capture the highest impact of these features along with
sqft\_living on house price.

~

\hypertarget{models---part-1}{%
\paragraph{Models - Part 1}\label{models---part-1}}

Our Base specification model is when we regress Price with Sqft\_living.
The Alternative specifications are shown below.

\begin{Shaded}
\begin{Highlighting}[]
\CommentTok{#renovated_factor and sqft_living}
\NormalTok{sqft_renovated <-}\StringTok{ }\KeywordTok{lm}\NormalTok{(price }\OperatorTok{~}\StringTok{ }\NormalTok{sqft_living }\OperatorTok{+}\StringTok{ }\KeywordTok{as.factor}\NormalTok{(renovated_factor) , }\DataTypeTok{data=}\NormalTok{data)}

\CommentTok{#Bathroom and sqft_living}
\NormalTok{sqft_bathrooms <-}\StringTok{  }\KeywordTok{lm}\NormalTok{(price }\OperatorTok{~}\StringTok{ }\NormalTok{sqft_living }\OperatorTok{+}\StringTok{ }\NormalTok{bathrooms , }\DataTypeTok{data=}\NormalTok{data)}

\CommentTok{#Bedrooms and sqft_living}
\NormalTok{sqft_bedrooms <-}\StringTok{  }\KeywordTok{lm}\NormalTok{(price }\OperatorTok{~}\StringTok{ }\NormalTok{sqft_living }\OperatorTok{+}\StringTok{ }\NormalTok{bedrooms , }\DataTypeTok{data=}\NormalTok{data)}

\CommentTok{#Waterfront and sqft_living}
\NormalTok{sqft_waterfront <-}\StringTok{  }\KeywordTok{lm}\NormalTok{(price }\OperatorTok{~}\StringTok{ }\NormalTok{sqft_living }\OperatorTok{+}\StringTok{ }\KeywordTok{as.factor}\NormalTok{(waterfront) , }\DataTypeTok{data=}\NormalTok{data)}

\CommentTok{#Sqft_above and sqft_living}
\NormalTok{sqft_sabove <-}\StringTok{ }\KeywordTok{lm}\NormalTok{((price) }\OperatorTok{~}\StringTok{ }\NormalTok{sqft_living }\OperatorTok{+}\StringTok{ }\NormalTok{sqft_above, }\DataTypeTok{data=}\NormalTok{data)}


\KeywordTok{stargazer}\NormalTok{(sqft_living,sqft_bathrooms,sqft_bedrooms,sqft_waterfront,sqft_sabove,sqft_renovated, }\DataTypeTok{se=}\KeywordTok{list}\NormalTok{(}\KeywordTok{cse}\NormalTok{(sqft_living),}\KeywordTok{cse}\NormalTok{(sqft_bathrooms), }\KeywordTok{cse}\NormalTok{(sqft_bedrooms),}\KeywordTok{cse}\NormalTok{(sqft_waterfront),}\KeywordTok{cse}\NormalTok{(sqft_sabove),}\KeywordTok{cse}\NormalTok{(sqft_renovated)),}\DataTypeTok{title=}\StringTok{"Price vs renovated_factor, Bedrooms,Bathrooms,waterfront,square_footage_above "}\NormalTok{, }\DataTypeTok{type=}\StringTok{"text"}\NormalTok{, }\DataTypeTok{star.cutoffs=}\OtherTok{NA}\NormalTok{, }\DataTypeTok{df=}\OtherTok{FALSE}\NormalTok{, }\DataTypeTok{digits=}\DecValTok{3}\NormalTok{)}
\end{Highlighting}
\end{Shaded}

\begin{verbatim}
## 
## Price vs renovatedBedrooms,Bathrooms,waterfront,square
## ==============================================================================================
##                                                     Dependent variable:                       
##                              -----------------------------------------------------------------
##                                                 price                     (price)     price   
##                                 (1)        (2)        (3)        (4)        (5)        (6)    
## ----------------------------------------------------------------------------------------------
## sqft_living                    0.281      0.284      0.317      0.273      0.295      0.279   
##                               (0.006)    (0.007)    (0.007)    (0.005)    (0.009)    (0.006)  
##                                                                                               
## bathrooms                                 -5.162                                              
##                                          (4.744)                                              
##                                                                                               
## bedrooms                                            -62.062                                   
##                                                     (3.542)                                   
##                                                                                               
## as.factor(waterfront)1                                         829.988                        
##                                                                (59.188)                       
##                                                                                               
## sqft_above                                                                 -0.019             
##                                                                           (0.007)             
##                                                                                               
## as.factor(renovated_factor)1                                                         159.957  
##                                                                                      (13.464) 
##                                                                                               
## Constant                      -43.603    -39.482     90.034    -32.981    -40.885    -46.352  
##                               (10.812)   (10.483)   (8.178)    (10.102)   (10.627)   (10.742) 
##                                                                                               
## ----------------------------------------------------------------------------------------------
## Observations                   21,612     21,612     21,612     21,612     21,612     21,612  
## R2                             0.493      0.493      0.508      0.531      0.493      0.501   
## Adjusted R2                    0.493      0.493      0.508      0.531      0.493      0.500   
## Residual Std. Error           261.454    261.447    257.482    251.515    261.353    259.477  
## F Statistic                  21,002.300 10,502.790 11,164.190 12,218.800 10,518.180 10,827.650
## ==============================================================================================
## Note:                                                                                       NA
\end{verbatim}

~

\hypertarget{correlations}{%
\paragraph{Correlations:}\label{correlations}}

\begin{Shaded}
\begin{Highlighting}[]
\CommentTok{#Bathroom and sqft_living}
\KeywordTok{paste}\NormalTok{(}\StringTok{"Correlation between bathrooms and Living Square footage: "}\NormalTok{, }\KeywordTok{round}\NormalTok{(}\KeywordTok{cor}\NormalTok{(data}\OperatorTok{$}\NormalTok{bathrooms, data}\OperatorTok{$}\NormalTok{sqft_living), }\DataTypeTok{digits=}\DecValTok{2}\NormalTok{))}
\end{Highlighting}
\end{Shaded}

\begin{verbatim}
## [1] "Correlation between bathrooms and Living Square footage:  0.75"
\end{verbatim}

\begin{Shaded}
\begin{Highlighting}[]
\CommentTok{#Bedrooms and sqft_living}
\KeywordTok{paste}\NormalTok{(}\StringTok{"Correlation between Bedrooms and Living Square footage: "}\NormalTok{, }\KeywordTok{round}\NormalTok{(}\KeywordTok{cor}\NormalTok{(data}\OperatorTok{$}\NormalTok{bedrooms, data}\OperatorTok{$}\NormalTok{sqft_living), }\DataTypeTok{digits=}\DecValTok{2}\NormalTok{))}
\end{Highlighting}
\end{Shaded}

\begin{verbatim}
## [1] "Correlation between Bedrooms and Living Square footage:  0.59"
\end{verbatim}

\begin{Shaded}
\begin{Highlighting}[]
\CommentTok{#Waterfront and sqft_living}
\KeywordTok{paste}\NormalTok{(}\StringTok{"Correlation between Waterfront and Living Square footage: "}\NormalTok{, }\KeywordTok{round}\NormalTok{(}\KeywordTok{cor}\NormalTok{(data}\OperatorTok{$}\NormalTok{waterfront, data}\OperatorTok{$}\NormalTok{sqft_living), }\DataTypeTok{digits=}\DecValTok{2}\NormalTok{))}
\end{Highlighting}
\end{Shaded}

\begin{verbatim}
## [1] "Correlation between Waterfront and Living Square footage:  0.1"
\end{verbatim}

\begin{Shaded}
\begin{Highlighting}[]
\CommentTok{#Sqft_above and sqft_living}
\KeywordTok{paste}\NormalTok{(}\StringTok{"Correlation between sqft_above and Living Square footage: "}\NormalTok{, }\KeywordTok{round}\NormalTok{(}\KeywordTok{cor}\NormalTok{(data}\OperatorTok{$}\NormalTok{sqft_above, data}\OperatorTok{$}\NormalTok{sqft_living), }\DataTypeTok{digits=}\DecValTok{2}\NormalTok{))}
\end{Highlighting}
\end{Shaded}

\begin{verbatim}
## [1] "Correlation between sqft_above and Living Square footage:  0.88"
\end{verbatim}

\begin{Shaded}
\begin{Highlighting}[]
\CommentTok{#renovated_factor and sqft_living}
\KeywordTok{paste}\NormalTok{(}\StringTok{"Correlation between renovated_factor and Living Square footage: "}\NormalTok{, }\KeywordTok{round}\NormalTok{(}\KeywordTok{cor}\NormalTok{(data}\OperatorTok{$}\NormalTok{renovated_factor, data}\OperatorTok{$}\NormalTok{sqft_living), }\DataTypeTok{digits=}\DecValTok{2}\NormalTok{))}
\end{Highlighting}
\end{Shaded}

\begin{verbatim}
## [1] "Correlation between renovated_factor and Living Square footage:  0.06"
\end{verbatim}

~

\begin{Shaded}
\begin{Highlighting}[]
\CommentTok{#Price vs bedroom and bathroom}
\CommentTok{#Bathroom and price}
\NormalTok{price_bathrooms <-}\StringTok{  }\KeywordTok{lm}\NormalTok{(price }\OperatorTok{~}\StringTok{ }\NormalTok{bathrooms , }\DataTypeTok{data=}\NormalTok{data)}

\CommentTok{#Bedrooms and price}
\NormalTok{price_bedrooms <-}\StringTok{  }\KeywordTok{lm}\NormalTok{(price }\OperatorTok{~}\StringTok{ }\NormalTok{bedrooms , }\DataTypeTok{data=}\NormalTok{data)}

\KeywordTok{stargazer}\NormalTok{(price_bathrooms,price_bedrooms, }\DataTypeTok{se=}\KeywordTok{list}\NormalTok{(}\KeywordTok{cse}\NormalTok{(price_bathrooms), }\KeywordTok{cse}\NormalTok{(price_bedrooms)),}\DataTypeTok{title=}\StringTok{"Price vs Bedrooms and Bathrooms"}\NormalTok{, }\DataTypeTok{type=}\StringTok{"text"}\NormalTok{, }\DataTypeTok{star.cutoffs=}\OtherTok{NA}\NormalTok{, }\DataTypeTok{df=}\OtherTok{FALSE}\NormalTok{, }\DataTypeTok{digits=}\DecValTok{3}\NormalTok{)}
\end{Highlighting}
\end{Shaded}

\begin{verbatim}
## 
## Price vs Bedrooms and Bathrooms
## ========================================
##                     Dependent variable: 
##                     --------------------
##                            price        
##                        (1)        (2)   
## ----------------------------------------
## bathrooms            250.332            
##                      (6.101)            
##                                         
## bedrooms                        127.548 
##                                 (3.608) 
##                                         
## Constant              10.688    110.316 
##                      (11.796)  (11.075) 
##                                         
## ----------------------------------------
## Observations          21,612    21,612  
## R2                    0.276      0.100  
## Adjusted R2           0.276      0.099  
## Residual Std. Error  312.443    348.399 
## F Statistic         8,228.977  2,387.925
## ========================================
## Note:                                 NA
\end{verbatim}

~

\begin{itemize}
\item
  When the number of bedrooms or bathrooms increases the price
  intuitively should increase but we don't see that in the first
  regression table. We then regressed bedrooms and bathrooms
  individually with price and they had a positive relationship. But when
  we add these variables in conjunction with sqft\_living, their
  coefficients are negative. This indicates that these models are
  suffering from imperfect multicollinearity and hence the variables
  bedrooms and bathrooms should not be included in the model.
\item
  When waterfront is included, the adjusted R2 increases. The
  coefficient of sqft\_living decreases. This means the model was
  suffering from upward omitted variable bias. Generally waterfront
  properties have higher sqft\_living and higher prices which explains
  the upward omitted variable bias. Therefore, we include waterfront as
  one of our control variables.
\item
  We see a similar problem as that of bedrooms and bathrooms, when
  sqft\_above is added. The standard error increases and as the
  correlation between sqft\_living and sqft\_living is 0.87 it is
  suffering from imperfect multicollinearity as well. Hence we exclude
  the variable sqft\_above from our final model.
\item
  When we include renovated\_factor in our model, the adjusted R2
  increases and the positive coefficient makes sense intuitively.
  i.e.~If the house is renovated, the price increase by \$160000. Also,
  the coefficient of the sqft\_living decreases, which means that the
  model was suffering from upward omitted variable bias and the variable
  renovated factor helps us correct for that and hence we decide to
  include the variable in our model.
\end{itemize}

~

\hypertarget{models---part-2}{%
\subsubsection{Models - Part 2}\label{models---part-2}}

Next we regressed Price with view, condition, grade, age\_of\_house and
yr\_renovated.

\begin{Shaded}
\begin{Highlighting}[]
\CommentTok{#Condition and Living Square footage}
\NormalTok{sqft_condition <-}\StringTok{  }\KeywordTok{lm}\NormalTok{(price }\OperatorTok{~}\StringTok{ }\NormalTok{sqft_living }\OperatorTok{+}\StringTok{ }\NormalTok{condition , }\DataTypeTok{data=}\NormalTok{data)}

\CommentTok{#Age of the house and Living Square footage}
\NormalTok{sqft_age_of_house <-}\StringTok{  }\KeywordTok{lm}\NormalTok{(price }\OperatorTok{~}\StringTok{ }\NormalTok{sqft_living }\OperatorTok{+}\StringTok{ }\NormalTok{age_of_house , }\DataTypeTok{data=}\NormalTok{data)}

\CommentTok{#Grade and Living Square footage}
\NormalTok{sqft_grade <-}\StringTok{  }\KeywordTok{lm}\NormalTok{(price }\OperatorTok{~}\StringTok{ }\NormalTok{sqft_living }\OperatorTok{+}\StringTok{ }\NormalTok{grade , }\DataTypeTok{data=}\NormalTok{data)}

\CommentTok{#View and Living Square footage}
\NormalTok{sqft_view <-}\StringTok{  }\KeywordTok{lm}\NormalTok{(price }\OperatorTok{~}\StringTok{ }\NormalTok{sqft_living }\OperatorTok{+}\StringTok{ }\KeywordTok{as.factor}\NormalTok{(view) , }\DataTypeTok{data=}\NormalTok{data)}

\CommentTok{#Floors and Living Square footage}
\NormalTok{sqft_floors <-}\StringTok{  }\KeywordTok{lm}\NormalTok{(price }\OperatorTok{~}\StringTok{ }\NormalTok{sqft_living }\OperatorTok{+}\StringTok{ }\NormalTok{floors , }\DataTypeTok{data=}\NormalTok{data)}


\KeywordTok{stargazer}\NormalTok{(sqft_living, sqft_condition,sqft_age_of_house,sqft_grade,sqft_view, sqft_floors,  }\DataTypeTok{se=}\KeywordTok{list}\NormalTok{(}\KeywordTok{cse}\NormalTok{(sqft_living),}\KeywordTok{cse}\NormalTok{(sqft_condition), }\KeywordTok{cse}\NormalTok{(sqft_age_of_house),}\KeywordTok{cse}\NormalTok{(sqft_grade),}\KeywordTok{cse}\NormalTok{(sqft_view), }\KeywordTok{cse}\NormalTok{(sqft_floors)),}

                     \DataTypeTok{title=}\StringTok{"Price vs Condition, Age of the House, Grade and View respectively "}\NormalTok{, }\DataTypeTok{type=}\StringTok{"text"}\NormalTok{, }\DataTypeTok{star.cutoffs=}\OtherTok{NA}\NormalTok{, }\DataTypeTok{df=}\OtherTok{FALSE}\NormalTok{, }\DataTypeTok{digits=}\DecValTok{3}\NormalTok{)}
\end{Highlighting}
\end{Shaded}

\begin{verbatim}
## 
## Price vs Condition, Age of the House, Grade and View respectively
## ====================================================================================
##                                           Dependent variable:                       
##                     ----------------------------------------------------------------
##                                                  price                              
##                        (1)        (2)        (3)        (4)        (5)       (6)    
## ------------------------------------------------------------------------------------
## sqft_living           0.281      0.282      0.305      0.184      0.256     0.279   
##                      (0.006)    (0.006)    (0.006)    (0.008)    (0.005)   (0.006)  
##                                                                                     
## condition                        43.908                                             
##                                 (2.810)                                             
##                                                                                     
## age_of_house                                2.354                                   
##                                            (0.085)                                  
##                                                                                     
## grade                                                  98.559                       
##                                                       (3.729)                       
##                                                                                     
## as.factor(view)1                                                 169.618            
##                                                                 (20.421)            
##                                                                                     
## as.factor(view)2                                                 127.664            
##                                                                 (11.784)            
##                                                                                     
## as.factor(view)3                                                 214.295            
##                                                                 (19.534)            
##                                                                                     
## as.factor(view)4                                                 620.884            
##                                                                 (39.226)            
##                                                                                     
## floors                                                                      6.475   
##                                                                            (3.890)  
##                                                                                     
## Constant             -43.603    -197.099   -208.701   -598.157   -14.469   -50.477  
##                      (10.812)   (15.558)   (14.716)   (17.748)  (10.310)   (9.423)  
##                                                                                     
## ------------------------------------------------------------------------------------
## Observations          21,612     21,612     21,612     21,612    21,612     21,612  
## R2                    0.493      0.499      0.525      0.535      0.544     0.493   
## Adjusted R2           0.493      0.499      0.525      0.534      0.544     0.493   
## Residual Std. Error  261.454    259.900    253.111    250.493    247.900   261.440  
## F Statistic         21,002.300 10,757.240 11,929.370 12,407.130 5,158.716 10,504.000
## ====================================================================================
## Note:                                                                             NA
\end{verbatim}

~

\hypertarget{correlations-1}{%
\paragraph{Correlations:}\label{correlations-1}}

\begin{Shaded}
\begin{Highlighting}[]
\CommentTok{#Condition and Living Square footage}
\KeywordTok{paste}\NormalTok{(}\StringTok{"Condition and Living Square footage: "}\NormalTok{, }\KeywordTok{round}\NormalTok{(}\KeywordTok{cor}\NormalTok{(data}\OperatorTok{$}\NormalTok{condition, data}\OperatorTok{$}\NormalTok{sqft_living), }\DataTypeTok{digits=}\DecValTok{2}\NormalTok{))}
\end{Highlighting}
\end{Shaded}

\begin{verbatim}
## [1] "Condition and Living Square footage:  -0.06"
\end{verbatim}

\begin{Shaded}
\begin{Highlighting}[]
\CommentTok{# Age of the house and Living Square footage}
\KeywordTok{paste}\NormalTok{(}\StringTok{"Year the age_of_house and Living Square footage: "}\NormalTok{, }\KeywordTok{round}\NormalTok{(}\KeywordTok{cor}\NormalTok{(data}\OperatorTok{$}\NormalTok{age_of_house, data}\OperatorTok{$}\NormalTok{sqft_living), }\DataTypeTok{digits=}\DecValTok{2}\NormalTok{))}
\end{Highlighting}
\end{Shaded}

\begin{verbatim}
## [1] "Year the age_of_house and Living Square footage:  -0.32"
\end{verbatim}

\begin{Shaded}
\begin{Highlighting}[]
\CommentTok{# View and Living Square footage}
\KeywordTok{paste}\NormalTok{(}\StringTok{"View and Living Square footage: "}\NormalTok{, }\KeywordTok{round}\NormalTok{(}\KeywordTok{cor}\NormalTok{(data}\OperatorTok{$}\NormalTok{view, data}\OperatorTok{$}\NormalTok{sqft_living), }\DataTypeTok{digits=}\DecValTok{2}\NormalTok{))}
\end{Highlighting}
\end{Shaded}

\begin{verbatim}
## [1] "View and Living Square footage:  0.28"
\end{verbatim}

\begin{Shaded}
\begin{Highlighting}[]
\CommentTok{# Floors and Living Square footage}
\KeywordTok{paste}\NormalTok{(}\StringTok{"Floors and Living Square footage: "}\NormalTok{, }\KeywordTok{round}\NormalTok{(}\KeywordTok{cor}\NormalTok{(data}\OperatorTok{$}\NormalTok{floors, data}\OperatorTok{$}\NormalTok{sqft_living), }\DataTypeTok{digits=}\DecValTok{2}\NormalTok{))}
\end{Highlighting}
\end{Shaded}

\begin{verbatim}
## [1] "Floors and Living Square footage:  0.35"
\end{verbatim}

\begin{itemize}
\item
  The correlation between condition with sqft\_living is very small (i.e
  -0.06). That means that the model was suffering from omitted variable
  bias which is not severe since the values are not significantly
  different from the base specification. Hence we decide to exclude this
  from our final model.
\item
  Age\_of\_house has a downward bias as the coefficient of sqft\_living
  increases and also it is negatively correlated with price. However
  even though the adjusted R2 of the model increases, it is counter
  intuitive because an older house should be cheaper and hence we decide
  to exclude age\_of\_house from our final model.
\item
  The Floors variable when added to the model isn't significant and the
  R2 does not increase. Therefore we will not add it in our model.
\item
  The addition of the other two variables grade and view with
  sqft\_living, indicates upward omitted variable bias has been removed
  since the coefficient of sqft\_living decreases and overall model
  performance increases. Hence we decide to keep grade and view in our
  final model.
\end{itemize}

~

\hypertarget{final-models}{%
\subsection{Final Models:}\label{final-models}}

\begin{Shaded}
\begin{Highlighting}[]
\CommentTok{#Model 1}
\NormalTok{sqft_living <-}\StringTok{ }\KeywordTok{lm}\NormalTok{(price }\OperatorTok{~}\StringTok{ }\NormalTok{sqft_living, }\DataTypeTok{data=}\NormalTok{data)}

\CommentTok{#Model2}
\NormalTok{sqft_waterfront <-}\StringTok{  }\KeywordTok{lm}\NormalTok{(price }\OperatorTok{~}\StringTok{ }\NormalTok{sqft_living }\OperatorTok{+}\StringTok{ }\KeywordTok{as.factor}\NormalTok{(waterfront), }\DataTypeTok{data=}\NormalTok{data)}

\CommentTok{#Model 3}
\NormalTok{sqft_waterfront_grade <-}\StringTok{  }\KeywordTok{lm}\NormalTok{(price }\OperatorTok{~}\StringTok{ }\NormalTok{sqft_living }\OperatorTok{+}\StringTok{ }\KeywordTok{as.factor}\NormalTok{(waterfront) }\OperatorTok{+}\StringTok{ }\NormalTok{grade , }\DataTypeTok{data=}\NormalTok{data)}

\CommentTok{#Model 4}
\NormalTok{sqft_waterfront_g_v <-}\StringTok{  }\KeywordTok{lm}\NormalTok{(price }\OperatorTok{~}\StringTok{ }\NormalTok{sqft_living }\OperatorTok{+}\StringTok{ }\KeywordTok{as.factor}\NormalTok{(waterfront) }\OperatorTok{+}\StringTok{ }\NormalTok{grade }\OperatorTok{+}\StringTok{ }\KeywordTok{as.factor}\NormalTok{(view) , }\DataTypeTok{data=}\NormalTok{data)}

\CommentTok{#Model 5}
\NormalTok{sqft_waterfront_g_v_r <-}\StringTok{  }\KeywordTok{lm}\NormalTok{(price }\OperatorTok{~}\StringTok{ }\NormalTok{sqft_living }\OperatorTok{+}\StringTok{ }\KeywordTok{as.factor}\NormalTok{(waterfront) }\OperatorTok{+}\StringTok{ }\NormalTok{renovated_factor }\OperatorTok{+}\StringTok{ }\NormalTok{grade }\OperatorTok{+}\StringTok{ }\KeywordTok{as.factor}\NormalTok{(view), }\DataTypeTok{data=}\NormalTok{data)}

\CommentTok{#Model 6}
\NormalTok{reg6 <-}\StringTok{ }\KeywordTok{lm}\NormalTok{(price }\OperatorTok{~}\StringTok{ }\NormalTok{sqft_living }\OperatorTok{+}\StringTok{ }\KeywordTok{as.factor}\NormalTok{(waterfront) }\OperatorTok{+}\StringTok{ }\NormalTok{grade }\OperatorTok{+}\StringTok{ }\NormalTok{renovated_factor }\OperatorTok{+}\StringTok{ }\KeywordTok{as.factor}\NormalTok{(view) }\OperatorTok{+}\NormalTok{floors}\OperatorTok{*}\NormalTok{renovated_factor, }\DataTypeTok{data=}\NormalTok{data)}


\CommentTok{#stargazer}
\KeywordTok{stargazer}\NormalTok{(sqft_living, sqft_waterfront, sqft_waterfront_grade, sqft_waterfront_g_v, sqft_waterfront_g_v_r, reg6, }\DataTypeTok{se=}\KeywordTok{list}\NormalTok{(}\KeywordTok{cse}\NormalTok{(sqft_living),}\KeywordTok{cse}\NormalTok{(sqft_waterfront),}\KeywordTok{cse}\NormalTok{(sqft_waterfront_grade), }\KeywordTok{cse}\NormalTok{(sqft_waterfront_g_v), }\KeywordTok{cse}\NormalTok{(sqft_waterfront_g_v_r), }\KeywordTok{cse}\NormalTok{(reg6)),}\DataTypeTok{title=}\StringTok{"Price vs Square living footage with control variables "}\NormalTok{, }\DataTypeTok{type=}\StringTok{"text"}\NormalTok{, }\DataTypeTok{star.cutoffs=}\OtherTok{NA}\NormalTok{, }\DataTypeTok{df=}\OtherTok{FALSE}\NormalTok{, }\DataTypeTok{digits=}\DecValTok{3}\NormalTok{)}
\end{Highlighting}
\end{Shaded}

\begin{verbatim}
## 
## Price vs Square living footage with control variables
## =====================================================================================
##                                              Dependent variable:                     
##                         -------------------------------------------------------------
##                                                     price                            
##                            (1)        (2)        (3)       (4)       (5)       (6)   
## -------------------------------------------------------------------------------------
## sqft_living               0.281      0.273      0.177     0.165     0.163     0.163  
##                          (0.006)    (0.005)    (0.008)   (0.008)   (0.008)   (0.007) 
##                                                                                      
## as.factor(waterfront)1              829.988    825.150   511.762   492.703   495.239 
##                                     (59.188)  (59.683)  (75.956)  (75.686)  (75.330) 
##                                                                                      
## renovated_factor                                                   129.925   -20.337 
##                                                                   (12.057)  (41.723) 
##                                                                                      
## grade                                          98.037    94.505    96.134    103.109 
##                                                (3.499)   (3.493)   (3.415)   (3.404) 
##                                                                                      
## as.factor(view)1                                         167.535   160.188   152.263 
##                                                         (19.534)  (19.262)  (19.261) 
##                                                                                      
## as.factor(view)2                                         111.457   107.162   100.936 
##                                                         (11.075)  (10.980)  (11.060) 
##                                                                                      
## as.factor(view)3                                         177.172   169.044   161.201 
##                                                         (18.917)  (18.645)  (18.575) 
##                                                                                      
## as.factor(view)4                                         383.768   374.421   363.556 
##                                                         (45.861)  (45.799)  (45.756) 
##                                                                                      
## floors                                                                       -34.632 
##                                                                              (3.506) 
##                                                                                      
## renovated_factor:floors                                                      100.213 
##                                                                             (30.713) 
##                                                                                      
## Constant                 -43.603    -32.981   -584.661  -548.597  -560.888  -561.926 
##                          (10.812)   (10.102)  (16.812)  (16.286)  (16.033)  (16.263) 
##                                                                                      
## -------------------------------------------------------------------------------------
## Observations              21,612     21,612    21,612    21,612    21,612    21,612  
## R2                        0.493      0.531      0.572     0.591     0.596     0.598  
## Adjusted R2               0.493      0.531      0.572     0.591     0.596     0.598  
## Residual Std. Error      261.454    251.515    240.223   234.799   233.369   232.697 
## F Statistic             21,002.300 12,218.800 9,623.096 4,461.779 3,985.364 3,219.449
## =====================================================================================
## Note:                                                                              NA
\end{verbatim}

For each category of dependent variable price, we found one variable to
control each aspect. For example, from size aspect, we just selected the
variable of interest sqft\_living in our model. We did not include any
of the remaining variables from the size category in our final model. If
we were to include more than one control variable to control for the
same category of our dependent variable, multicollinearity becomes a
problem which would throw off our coefficient estimates.To avoid that,
we keep \textbf{regression 5 (sqft\_waterfront\_g\_v\_r)} as our final
model. The variables included in the model from each category are:

\textbf{Grade:} because we feel grade or the quality of construction for
the house would play a role in how low or high the price of the house
would be. If the house has better construction, meaning it has better
plumbing, roofing or newer appliances the house would be more expensive.
Similarily if the house had a lower Grade the house would be less
expensive. Overall,as grade increases it has a positive relationship
with price.

\textbf{Waterfront and View:} Houses with better views are expected to
be more expensive. The view of a house is also indicative of location.
In our case, houses near the water are more expensive and have better
views compared to houses more inland. The view variable accounts for two
things and explains house price increases or decreases well which is why
it is one of our control variables. Overall,as view increases it has a
positive relationship with price.

And lastly, the \textbf{dummy variable Renovation} indicates if any
remodelling was done. Renovations are known to increase the value of
older homes and allow them to potentially be priced at a higher rate. If
Renovation is 1 it causes an increase in price.

Each of these variables are included in the model and we see that,
estimated beta hat of sqft\_living is suffering from upward omitted
variable bias and by adding these variables we are able to fix that
bias. The coefficient estimate of sqft\_living decreases as expected.
And the model performance increases as we move from model 1 to 5.

\hypertarget{interpretation-for-each-of-the-variable-in-the-final-model-includes}{%
\paragraph{\texorpdfstring{\textbf{Interpretation for each of the
variable in the final model
includes:}}{Interpretation for each of the variable in the final model includes:}}\label{interpretation-for-each-of-the-variable-in-the-final-model-includes}}

\emph{sqft\_living}

\begin{itemize}
\item
  Estimate : 0.163
\item
  Interpretation: Holding everything else constant, the marginal effect
  of 1 foot increase in sqft\_living increases the house price on an
  average by \$163.
\end{itemize}

\emph{waterfront}

\begin{itemize}
\item
  Estimate : 492.703
\item
  Interpretation: Holding everything else constant, a house with a
  waterfront has house price higher by \$492703 on an average as
  compared to houses without waterfront.
\end{itemize}

\emph{renovated\_factor}

\begin{itemize}
\item
  Estimate : 129.925
\item
  Interpretation: Holding everything else constant, a house which is
  renovated increases the price on average by \$129925 as compared to a
  house that is not renovated.
\end{itemize}

\emph{grade}

\begin{itemize}
\item
  Estimate : 96.134
\item
  Interpretation: Holding everything else constant, with each level
  increase in grade (i.e.~from poor building construction and design to
  good building construction and design), the price increases by \$96134
  on an average.
\end{itemize}

\emph{view1}

\begin{itemize}
\item
  Estimate : 167.535
\item
  Interpretation: Holding everything else constant, if the house has
  view 1 on a scale of 0-4, the price increases by \$167535 on an
  average.
\end{itemize}

\emph{view2}

\begin{itemize}
\item
  Estimate : 107.162
\item
  Interpretation: Holding everything else constant, if the house has
  view 2 on a scale of 0-4, the price increases by \$107162 on an
  average.
\end{itemize}

\emph{view3}

\begin{itemize}
\item
  Estimate : 169.044
\item
  Interpretation: Holding everything else constant, if the house has
  view 3 on a scale of 0-4, the price increases by \$169044 on an
  average.
\end{itemize}

\emph{view4}

\begin{itemize}
\item
  Estimate : 374.421
\item
  Interpretation: Holding everything else constant, if the house has
  view 4 on a scale of 0-4, the price increases by \$374421 on an
  average.
\end{itemize}

\begin{Shaded}
\begin{Highlighting}[]
\CommentTok{#Calculating the correlation between control variables and variable of interest(sqft_living)}

\KeywordTok{round}\NormalTok{(}\KeywordTok{cor}\NormalTok{(data}\OperatorTok{$}\NormalTok{sqft_living,data[}\KeywordTok{c}\NormalTok{(}\DecValTok{7}\NormalTok{,}\DecValTok{8}\NormalTok{,}\DecValTok{10}\NormalTok{,}\DecValTok{20}\NormalTok{)]),}\DecValTok{2}\NormalTok{)}
\end{Highlighting}
\end{Shaded}

\begin{verbatim}
##      waterfront view grade renovated_factor
## [1,]        0.1 0.28  0.76             0.06
\end{verbatim}

Another reason we have chosen to add these control variables along with
our variable of interest is their correlation. A renovated home could
mean anything from new appliances to expanding the actual sqft\_living.
With the potentially of meaningfully impacting the variable of interest
we added the dummy variable to the model. Next the waterfront variable
has a subtle relationship with living footage. Homes near the water are
known to be more expensive for a multitude of reasons one of which being
that the homes are expected to be larger. View is linked a little with
waterfront as both of these variables take into consideration the
location. As expected certain neighborhoods have larger homes and just
in general more sqft\_living. Lastly, grade is associated with
construction which accounts for things like closet areas and stairways.
Things such as enclosed porches are sometimes included in sqft\_living
which is associated with a higher grade value. Overall, these four
control variables play a role in determining price and also have a
relationship with our variable of interest.

\textbf{We also validate our final model findings by running t and
f-tests as follows:}

~

\hypertarget{testing-for-significance-of-our-control-variables-in-model-5-using-a-t-test}{%
\paragraph{\texorpdfstring{\textbf{Testing for significance of our
control variables in model 5 using a
T-test}}{Testing for significance of our control variables in model 5 using a T-test}}\label{testing-for-significance-of-our-control-variables-in-model-5-using-a-t-test}}

\begin{itemize}
\item
  sqft\_living : \textbar20.38\textbar{} \textgreater{} 1.96
\item
  as.factor(waterfront)1 : \textbar6.50\textbar{} \textgreater{} 1.96
\item
  renovated\_factor : \textbar10.78\textbar{} \textgreater{} 1.96
\item
  grade : \textbar28.15\textbar{} \textgreater{} 1.96
\item
  as.factor(view)1 : \textbar8.32\textbar{} \textgreater{} 1.96
\item
  as.factor(view)2 : \textbar9.76\textbar{} \textgreater{} 1.96
\item
  as.factor(view)3 : \textbar9.07\textbar{} \textgreater{} 1.96
\item
  as.factor(view)4 : \textbar8.18\textbar{} \textgreater{} 1.96
\end{itemize}

We can see that all the control variables in our final model are
statistically significant at 5\% significance level
(i.e.~\textbar t-stat\textbar{} \textgreater{} 1.96). Therefore, we can
safely reject the null hypothesis that any of these variables do not
have a coefficient of zero.

~

\hypertarget{performing-f-tests-to-determine-the-significance-of-the-model}{%
\paragraph{Performing F tests to determine the significance of the
model}\label{performing-f-tests-to-determine-the-significance-of-the-model}}

\begin{Shaded}
\begin{Highlighting}[]
\CommentTok{### Comparing model with waterfront, grade and view i.e. model 5 with model 1}

\NormalTok{unrestricted_model <-}\StringTok{ }\NormalTok{sqft_waterfront_g_v_r}
\NormalTok{restricted_model <-}\StringTok{ }\NormalTok{sqft_living}
\KeywordTok{anova}\NormalTok{(restricted_model, unrestricted_model)}
\end{Highlighting}
\end{Shaded}

\begin{verbatim}
## Analysis of Variance Table
## 
## Model 1: price ~ sqft_living
## Model 2: price ~ sqft_living + as.factor(waterfront) + renovated_factor + 
##     grade + as.factor(view)
##   Res.Df        RSS Df Sum of Sq      F    Pr(>F)    
## 1  21610 1477223932                                  
## 2  21603 1176525231  7 300698700 788.76 < 2.2e-16 ***
## ---
## Signif. codes:  0 '***' 0.001 '**' 0.01 '*' 0.05 '.' 0.1 ' ' 1
\end{verbatim}

Comparing the Restricted Model with the Unrestricted Model, we can see
that the Unrestricted model is a better fit as its p value is less than
0.05 and the F value ie greater than the critical value. Therefore we
reject the null hypothesis and state that at least one of the
coefficients of the variables is non-zero

\begin{Shaded}
\begin{Highlighting}[]
\CommentTok{### Comparing model 6 with model 5(our final model)}
\NormalTok{unrestricted_model_}\DecValTok{1}\NormalTok{ <-}\StringTok{ }\NormalTok{reg6}
\NormalTok{restricted_model_}\DecValTok{1}\NormalTok{ <-}\StringTok{ }\NormalTok{sqft_waterfront_g_v_r}
\KeywordTok{anova}\NormalTok{(restricted_model_}\DecValTok{1}\NormalTok{, unrestricted_model_}\DecValTok{1}\NormalTok{)}
\end{Highlighting}
\end{Shaded}

\begin{verbatim}
## Analysis of Variance Table
## 
## Model 1: price ~ sqft_living + as.factor(waterfront) + renovated_factor + 
##     grade + as.factor(view)
## Model 2: price ~ sqft_living + as.factor(waterfront) + grade + renovated_factor + 
##     as.factor(view) + floors * renovated_factor
##   Res.Df        RSS Df Sum of Sq      F    Pr(>F)    
## 1  21603 1176525231                                  
## 2  21601 1169646389  2   6878842 63.519 < 2.2e-16 ***
## ---
## Signif. codes:  0 '***' 0.001 '**' 0.01 '*' 0.05 '.' 0.1 ' ' 1
\end{verbatim}

Comparing model 6 and 5, from the above regression table we can see that
the beta hat estimator is not changing. As the beta hat has converged we
can say that we have gotten our final model. The addition of the
interaction variable floors with renovated\_factor improves the goodness
of fit slightly. We will stick with regression 5 as our final
modelbecause floors variable doesn't add much value even though the
R-square increases and also because we do not want to overfit the model.
We believe that the control variables we have chosen take into
consideration a variety of factors that impact house price as a whole.

~

\hypertarget{conclusion}{%
\subsection{Conclusion:}\label{conclusion}}

The Square Foot Living of the house has a positive impact on house
prices. Furthermore, as new amenities are added to a house, the house
price will increase.

~

\emph{Internal Validity} Our estimators were unbiased and consistent
because as the sample size increases the estimates are getting closer to
the true value. Also the estimators are converging to the true value.

\begin{itemize}
\item
  Furthermore, we corrected for Heteroskedasticity by using the cse
  function which means that the first assumption of the least square
  i.e.~Var(x\textbar u) = 0 is met.
\item
  The second assumption is met because our variable of interest and
  dependent variable are i.i.d which is assumed for cross-sectional
  observational data.
\item
  The third assumption is met, as we determined which outliers could be
  kept and which needed to be removed.
\item
  For the fourth assumption, we removed any control variables that
  resulted in Multicollinearity.
\item
  As we have met the four assumptions for Least Squares we can say that
  our estimator is unbiased. E(hatβ0) =β0 and E(hatβ1) =β1.
\item
  Also comparing our restricted model to our unrestricted model our
  R\^{}2 and adjusted R\^{}2 increased from 49\% to 60\%. This means our
  final model explains 11\% more of the variation in Y than our base
  specification model.
\end{itemize}

~

\emph{External Validity} As our data is only house sales for one year in
King's County, we think we would need a larger sample size to generalize
our findings.

\end{document}
